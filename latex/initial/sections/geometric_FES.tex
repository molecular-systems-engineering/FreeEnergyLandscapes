\subsection{Position-dependent compression of configuration space: Geometric Free Energy Surface}
\label{sec:Geometric}
When a free energy profile is expressed along a curvilinear reaction coordinate, geometric contributions naturally appear. For instance, in the case of a distance coordinate, the probability density scales with the measure of the corresponding hyperspherical shell, leading to an entropic term in the associated potential of mean force. This contribution is not specific to any interaction but arises purely from the geometry of configuration space. It reflects the fact that a single value of a coordinate might not necessarily correspond to a macroscopic, identifiable state. 

\subsubsection{Potential of Mean Force and the role of the metric}
The potential of mean force\cite{onsager1933theories,kirkwood_statistical_1949} formalizes the idea that an effective two-body potential could describe many-body correlations averaged over solvent and other particles. The probability density \(P(\xi)\) of a RC \(\xi(\mathbf r)\) to have a specific value \(\xi\) is 
\begin{equation}
p(\xi)=\frac{1}{Q}\int d\mathbf r\,\delta\big(\xi(\mathbf r)-\xi\big)e^{-\beta U(\mathbf r)},
\end{equation}
and the PMF with respect to a reference \(\xi_0\) is
\begin{equation}
w(\xi)=-kT\ln P(\xi)+w(\xi_0)   
\end{equation}

\subsubsection{Intuitive view:} when the potential \(U=0\), the PMF should be uniform and the probability of finding the system in a region of configuration space must be proportional to the accessible volume. For the distance \(r\) between two particles, the probability scales with the volume of the spherical shell \(4\pi r^{2} dr\). This motivates the definition of the PMF from the radial probability density,
\begin{equation}
P(r) \propto 4\pi r^{2}e^{-\beta w(r)},
\qquad
w(r)=-kT\ln \left[ P(r)/ 4 \pi r^2\right] +\mathrm{const}.
\end{equation}
The extra term \(kT\ln(4\pi r^2)\) is therefore entropic, reflecting the growing number of configurations at larger separations.

\subsubsection{Which probability?}
One may ask whether to define probabilities directly in terms of \(w\) or to include geometric factors ``by hand''. The rigorous answer is that the correct measure is determined by marginalizing the full phase–space density.
In generalized coordinates \(\mathbf{q}\) with conjugate momenta \(\mathbf{p}\), the canonical distribution is~\cite{gibbs1906scientific}
\begin{equation}
P(\mathbf{q},\mathbf{p})\propto e^{-\beta \left[ \frac{1}{2} \mathbf{p}^t M^{-1}(\mathbf{q})\mathbf{p} + U(\mathbf{q})\right]},
\end{equation}
with the mass–metric tensor \(M(\mathbf{q})=J(\mathbf{q})^t m J(\mathbf{q})\), where \(J\) is the Jacobian of the transformation from Cartesian to generalized coordinates and \(m\) the diagonal matrix with the atomic masses.
Integrating out momenta via a Gaussian integral gives
\begin{equation}
\int d\mathbf{p} e^{-\frac{1}{2}\beta \mathbf{p}^tM^{-1}\mathbf{p}}
= (2\pi kT)^{n/2}\sqrt{\det M(\mathbf{q})},
\end{equation}
so that the configurational probability is
\begin{equation}
P(\mathbf{q})\propto \sqrt{\det M(\mathbf{q}) } e^{-\beta U(\mathbf{q})}.
\end{equation}
If the masses are all equal, they factorize out, and instead of \(M\), the metric factor \(g = J_\mathbf{q}^t J_\mathbf{q}\) is used, where \(\sqrt{\det g} = \vol{J_\mathbf{q}}\) is the volume element.  For spherical coordinates of a relative vector,
\( \sqrt{\det g}=r^{2}\sin\theta\), and for an isotropic environment, integrating over the solid angle one recovers the intuitive result.
