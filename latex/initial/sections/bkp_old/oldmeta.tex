

A powerful class of enhanced sampling approaches works by applying an explicitly time- or history-dependent biasing potential along some collective variable(s) ($\boldsymbol{\xi}(r)$), with the dual aim of accelerating rare transitions and recovering the unbiased distribution ($p(\boldsymbol{\xi})$) by proper reweighting. 



such that barriers in the effective free energy ($F(\boldsymbol{\xi})$) are reduced and transitions are promoted. The fundamental requirement is that one can reweight from the biased trajectory back to the canonical ensemble to recover
\begin{equation}
p_0(\boldsymbol{\xi}) \propto p_V(\boldsymbol{\xi}) e^{+\beta V(\boldsymbol{\xi})}
\end{equation}
where \(p_V\) is the marginal histogram of $\boldsymbol{\xi}$ under the bias.
Umbrella sampling is arguably the prototypical method based on bias. In a family of windowed simulations, one imposes a static harmonic restraint \(V_i(\boldsymbol{\xi}) = \frac{1}{2} k_i (\boldsymbol{\xi} - \boldsymbol{\xi}_i^0)^2\). Each window samples the local region around $\boldsymbol{\xi}_i^0$, producing a biased histogram $p_{i,V}(\boldsymbol{\xi})$. To combine windows, one uses the standard Zwanzig free-energy perturbation style reweighting (see Sec.\ref{sec:FEP}) 
\begin{equation}
\Delta F_{i\to j} = -k_{\rm B} T \ln \Big\langle e^{-\beta [V_j(\boldsymbol{\xi}) - V_i(\boldsymbol{\xi})]} \Big\rangle_{i,V}
\end{equation}
and/or one uses WHAM / MBAR approaches to merge all the windows self-consistently into a global unbiased ($p_0(\boldsymbol{\xi})$). This reconstructs ($F(\boldsymbol{\xi})$) over a broad range of ($\boldsymbol{\xi}$). 
Umbrella sampling is robust, conceptually simple, and effective when one can define overlapping windows and suitable force constants; however, it is inherently non-adaptive and often inefficient in high-barrier or multi-dimensional landscapes.
Metadynamics builds on the same bias-potential philosophy but evolves ($V(\boldsymbol{\xi})$) on the fly. In its original flavor, small Gaussian increments ($\delta V = w \exp[-(\boldsymbol{\xi} - \boldsymbol{\xi}_t)^2/2\sigma^2]$) are deposited along the trajectory, gradually “filling” the visited free energy wells (the “computational sand” metaphor). in the long-time limit the bias accumulates to approximate (-$F(\boldsymbol{\xi})$), thereby flattening the effective free energy surface and allowing diffusive motion in ($\boldsymbol{\xi}$)-space. In well-tempered metadynamics (WTmetaD), the height of deposited Gaussians is tempered over time so that ($V(\boldsymbol{\xi})$) converges smoothly; the equilibrium result obeys ($V(\boldsymbol{\xi}) \approx - \frac{1}{\gamma - 1} F(\boldsymbol{\xi})$), where ($\gamma$) is the bias factor. %The time-dependent bias must be reweighted (e.g. via the Tiwary–Parrinello estimator or other methods) to recover the unbiased free energy.
While metadynamics is powerful, it can suffer from slow convergence and poor performance when the CVs are suboptimal. To address this, the On-the-fly Probability Enhanced Sampling (OPES) method was developed, combining elements of metadynamics and variationally enhanced sampling (VES) into a more controlled and rapidly converging biasing scheme. OPES maintains an estimate ($\hat p(\boldsymbol{\xi})$) of the target marginal probability distribution and updates ($V(\boldsymbol{\xi})$) such that
\begin{equation}
p_V(\boldsymbol{\xi}) \approx \hat p(\boldsymbol{\xi})
\end{equation}
i.e. the bias aims to flatten the sampled ($\boldsymbol{\xi}$)-distribution to the chosen target. OPES can be seen as a generalization of metadynamics in which the bias is computed from an on-the-fly reweighted probability rather than Gaussians alone. OPES typically converges faster (i.e., with less “overshooting”) and exhibits an explicit reweighting scheme that is stable even in multiple dimensions.