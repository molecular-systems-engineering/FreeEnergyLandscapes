\subsection{Free Energy Profiles and Equilibrium Constants}
Through the link between probabilities and partition functions, free energy profiles and the associated PMFs can be used to compute equilibrium constants for binding reactions. One must be careful when connecting to standard free energies of reaction, since the equilibrium constants defined in statistical mechanics depend on the reference concentration. Experimental \(\Delta G^\circ\) values are typically referring to the standard concentration of \(c^\circ=1\) M.

For an association reaction of the kind \(R+L \rightleftharpoons RL\) the law of mass action at low concentrations gives
\begin{equation}
K_{\rm eq}=\frac{[RL]}{[R][L]},\qquad
\Delta G_{\rm bind}^\circ=-k_BT\ln\left(K_{\rm eq}c^\circ\right).  
\end{equation}
Here we use \(\Delta G_{\rm bind}^\circ\) to denote the standard free energy of binding, following the convention in alchemical binding studies. The law of mass action formally defines the binding equilibrium constant above. Still, this definition is valid in the thermodynamic limit only, and cannot be applied directly in molecular simulations, which typically contain a single receptor and ligand in a finite box\cite{de2011determining}. A statistical–mechanical route to express the equilibrium constant that is more appropriate for small systems is using the ratio of Boltzmann probabilities for bound versus unbound configurations, 
\begin{equation}
\Delta G^0_{\mathrm{bind}} \approx \Delta A^0_{\mathrm{bind}} = -kT \ln \frac{P(RL)}{P(R+L)} - kT\ln \left(c^0/c\right),\label{eq:deltaG_shirts}
\end{equation}
where the effect of (typically small) volume changes has been neglected, and the term \(c^0/c\) converts from the molar concentration in the simulation box, \(c\), to the standard state.
 
If \(\Omega_{\rm site}\) denotes the binding region and \(\Omega_{\rm bulk}\) the unbound region, then 
\begin{equation}
\frac{P(RL)}{P(R+L)}  = \frac{\int_{\Omega_{\rm site}} e^{-\beta U(\mathbf q)} d\mathbf q }{\int_{\Omega_{\rm bulk}} e^{-\beta U(\mathbf q)} d\mathbf q} = \frac{Q_{RL}}{Q_{R+L}}
\end{equation}
and the (dimensional) equilibrium constant is then written in terms of probabilities and the simulation box volume as 
\begin{equation}
K_{\rm eq}=V\frac{Q_{RL}}{Q_{R+L}}\label{eq:k-shirts}
\end{equation}
A similar, equivalent, form is used by Roux, who writes:
\begin{equation}
K_{\rm eq}=
\frac{\int_{\Omega_{\rm site}} e^{-\beta U(\mathbf q)} d\mathbf q }{\int_{\Omega_{\rm bulk}} \delta(\mathbf     q - \mathbf q^*) e^{-\beta U(\mathbf q)} d\mathbf q}, 
\end{equation}
where the delta, which pins the ligand in the bulk, would yield the \(V\) factor in Eq.\ref{eq:k-shirts} once integrated if isotropy and homogeneity can be assumed in the bulk. As binding affinity simulations are typically performed using a series of restraints, their effect has to be carefully removed by unbiasing the results\cite{woo2005calculation,roux2008comment}.


Gilson and coworkers derived the same constant directly from activities for a rigid ligand
\begin{equation}
\Delta G^\circ_\mathrm{bind} = -RT \ln\left( \frac{c^\circ}{8\pi^2}\frac{\sigma_R \sigma_L}{\sigma_{RL}} \frac{Q_{RL},Q_S}{Q_{R},Z_{R}} \right)+ P^\circ \Delta \bar V_{AB}.
\end{equation}
where the factor \(8 \pi^2\) comes from the integration of momenta (constraints are not unbiased in this picture), the symmetry numbers \(\sigma_i\) take into account the degeneracy of configurations, and \(Q_S\) is the configurational partition function of the solvent. 
Here, the contribution coming from volume changes \(\Delta \bar V_{LR} = V_{LR} -V_L -V_R\) is spelled out explicitly. The connection with Eq.\ref{eq:deltaG_shirts} comes from the infinite dilution identity \(Q_RQ_L/ Q_S = Q_{R+L}\).

This formulation serves as the basis for the double–decoupling method (DDM), in which the ratio of partition functions is obtained by decoupling the ligand in the binding site and in the bulk solvent. Because simulations in the bound state require restraining the ligand’s position and orientation relative to the receptor, one must add back the analytical free energy cost of these restraints. Boresch and co-workers introduced a minimal and non-redundant set of six relative restraints (one distance $r$, two bond angles $\theta_i$, and three torsional $\phi_i$, for which the correction has a closed form,
\begin{equation}
\Delta G_{\rm restr} = -k T \ln \left[\frac{8\pi^2 V \sqrt{K_r K_{\theta_1} K_{\theta_2} K_{\phi_1} K_{\phi_2} K_{\phi_3}}}{r_0^2 \sin\theta_{1,0}\sin\theta_{2,0} , (2\pi kT)^3} \right],
\end{equation}
where \(r_0, \theta_{i,0}\) are the equilibrium restraint values and \(K_i\) are the force constants. The standard binding free energy has to be corrected by a similar term for reference values\cite{boresch2003absolute}. 

Various formulations of the DDM method rest on the same principle: the binding free energy is the difference between decoupling the ligand in bulk and in the binding site. In this way, DDM should be viewed less as a single protocol than as a general framework, encompassing related alchemical cycles such as confine-and-release. The central lesson is that restraints and their proper unbiasing are essential to obtain rigorous and transferable results, which is why DDM remains the standard route to absolute binding free energies.\cite{gilson2007calculation,bian2025formally}