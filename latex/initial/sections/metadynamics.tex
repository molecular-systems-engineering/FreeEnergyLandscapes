\subsection{Adaptive Bias Methods: Metadynamics}

Metadynamics (MetaD), introduced by Laio and Parrinello \cite{laio2002escaping}, is one of the most influential adaptive-bias algorithms for reconstructing FESs \cite{metaD_2,metaD_1,bussi2020using,valsson2016enhancing}.  Its central idea is to discourage a molecular simulation from revisiting previously explored regions of the space of CVs $\xi$; in doing so, metadynamics enhances the fluctuations along $\xi$, thus speeding up the sampling \cite{valsson2016enhancing}.  A MetaD simulation evolves under a time-dependent bias potential $V(\xi,t)$ that is incrementally constructed as the trajectory progresses.  At regular time intervals, a small repulsive Gaussian hill of height $w$ and width $\sigma$ is deposited at the instantaneous CV value $\xi(t)$:
\begin{equation}
V(\xi,t) = \sum_{t'<t} w
e^{-\frac{\left(\xi-\xi(t')\right)^2}{2\sigma^2 }}
\end{equation}
This ``computational sand-filling'' \cite{metaD_2} progressively raises the free-energy of ensembles of configurations visited during sampling, allowing the system to escape local minima and visit new regions of phase space.  In the long-time limit, the accumulated bias offsets the underlying free-energy surface $F(\xi)$ up to an additive constant, so that $V(\xi,t \to \infty)\approx -F(\xi)$.  
Once this condition is reached, the biased dynamics samples a uniform probability distribution in CV space, effectively restoring ergodicity (see Sec. \ref{sec:Theory}).

MetaD is conceptually related to other history-dependent approaches, such as the local elevation method \cite{localelevation1994}. In particular, MetaD shares with local elevation the general principle of discouraging revisits to previously explored regions of collective variable space, while differing in its formulation and bias-update protocol. Moreover, compared with earlier mean-force-based schemes such as the Adaptive Biasing Force (ABF) method \cite{abf_2015,abf_darve2001,abf_darve2002}, metadynamics constructs the bias from local visitation history rather than explicit force estimates. 

Despite its simplicity and general applicability, MetaD suffers from a few drawbacks: the continual deposition of Gaussians can lead to systematic overshooting of the free-energy surface; convergence depends on the choice of Gaussian height and width; and the time dependence of $V(\xi,t)$ complicates rigorous reweighting. These issues motivated the development of statistically controlled variants that address these shortcomings \cite{metaD_2,valsson2016enhancing,bussi2020using}. 


\begin{textbox}[t]
\subsubsection{Free-energy landscape of CO reduction on copper\label{sec:case-study-reduction}}
An excellent example of the use of metadynamics to discover reaction landscapes is the work by  Cheng, Xiao, and Goddard III\cite{cheng2017full} on the mechanism of CO reduction on copper. 
Copper remains the only elemental catalyst capable of reducing CO$_2$ into hydrocarbons at significant rates, but its product distribution and mechanistic pathways have long been debated. Using ab initio molecular dynamics with explicit water layers, combined with metadynamics and refined using the Blue Moon ensemble, the authors computed atomistic free-energy barriers and pathways. The analysis revealed that at moderate potentials (U > –0.6 V vs RHE, pH 7), ethylene is the dominant product, formed via CO–CO coupling in an Eley–Rideal pathway with water as the proton source (\(\Delta G^\ddag = 0.69\) eV). At more negative potentials, hydrogen competes for surface sites, suppressing C–C coupling and enabling methane formation, with \(^*\)CHO identified as the key intermediate.

The impact of this study is twofold. First, it demonstrated how explicit solvation and constant-potential modeling resolve longstanding discrepancies in previous DFT work, where implicit solvation gave inconsistent barriers. Second, by quantitatively reproducing the observed potential- and pH-dependent product distribution, it suggested a mechanism for for tuning selectivity in electrochemical CO$_2$ utilization, influencing subsequent efforts to design Cu-based alloys.

%\begin{figure}
{\centering  
\includegraphics[width=\linewidth]{images/goddard.pdf}
}
    %\caption{
    \textit{
    Free-energy landscape for ethylene formation on Cu(100). (a) snapshots from ab-initio simulations with explicit solvent revealed (b) that the Eley–Rideal mechanism (black) has consistently lower barriers than Langmuir–Hinshelwood (blue), establishing CO dimerization as the rate-determining step.}
    %\label{fig:goddard} \end{figure}
\end{textbox}

\subsubsection{Well-Tempered Metadynamics (WTMetaD)}
Well-tempered metadynamics \cite{wt_metaD_1} introduces a smooth tempering of the bias deposition rate to achieve self-limiting convergence. In WTMetaD the Gaussian height decreases exponentially with the local value of the bias already accumulated:
\begin{equation}
w(t) = w_0 e^\frac{-\beta{V(\xi(t),t)}}{(\gamma-1)}
\end{equation}

where $\gamma = (T + \Delta T)/T > 1$ is the \emph{bias factor}.  At the beginning of a simulation, Gaussians start with height $w_0$; as $V(\xi,t)$ grows, the added bias diminishes, so that (V) asymptotically approaches a fraction of the underlying free energy:
\begin{equation}
V(\xi,t\to\infty) = -({\gamma-1})^{-1} F(\xi).
\end{equation}

The bias factor $\gamma$ tunes the trade-off between exploration (large $\gamma$) and accuracy (small $\gamma$).
The stationary distribution sampled by WTMetaD in the long-time limit is no longer flat but \emph{well-tempered}, i.e. $p(\xi)\propto {e^{-\beta F(\xi)/\gamma}}$.  The sampling along $\xi$ therefore occurs at an effectively elevated temperature $T_\text{eff}=\gamma T$, which thus enhances barrier crossing while maintaining a known, analytically recoverable bias.
WTMetaD has, \emph{de facto}, become the standard formulation implemented in modern software (e.g., PLUMED \cite{plumed214}, GROMACS \cite{Lindahl2022}, LAMMPS \cite{Thompson2022}) because it provides controlled convergence, improved statistical efficiency, and the possibility to monitor the flattening of the free-energy landscape on-the-fly.

One can combine WTMetaD with static biases (e.g., harmonic restraints, walls, or custom static biases) in a straightforward, additive manner \cite{Awasthi2016, limongelli2013funnel, bjola2024estimating}. If the WTMEtaD bias is deposited sufficiently slowly - so that transition states remain effectively bias-free - the infrequent metadynamics framework allows barrier-crossing times to be rescaled to recover physical rate constants \cite{tiwary2013metadynamics,salvalaglio2014assessing,palacio2022transition}.


\begin{textbox}[t!]
\subsubsection{Choosing the right coordinate: the ring puckering example}
%\hspace*{-0.20\linewidth}
{\centering \includegraphics[width=1\linewidth]{Figures/Figure_puckering_2.png}}
The choice of collective variables (CVs) is critical in free-energy calculations, but not always obvious. 
Puckered ring conformers can be described by the Cremer--Pople cartesian coordinates, obtained from the out-of-plane displacements $z_j$ of the 6-membered ring atoms as\\
\[ q_x =  \sqrt{\frac{1}{3}} \sum_{j=1}^6 z_j \cos\left[\frac{2\pi}{3}(j-1)\right],\quad  q_y = -\sqrt{\frac{1}{3}} \sum_{j=1}^6 z_j \sin\left[\frac{2\pi}{3}(j-1)\right],\quad q_z = \sqrt{\frac{1}{6}} \sum_{j=1}^6 (-1)^{j-1} z_j,\]
or by their polar representation 
\(\left(Q \sin \theta \cos \phi, Q \sin \theta \sin \phi, Q\cos \theta\right)\).
These coordinates correspond to a discrete Fourier decomposition of the atomic elevations over the mean molecular plane, with the angular coordinates spanning all pseudorotations (middle panel, where $\theta=0$ and $\pi$ correspond to the chair and inverted chair conformations, respectively). Only the coordinates $(\theta,\phi)$ turn out to be useful biasing variables, as they control connectivity between conformers. The left panel shows the fully sampled puckering free energy landscape of glucuronic acid from a metadynamics run using $(\theta,\phi)$ as CVs (5 kJ/mol isolines). In contrast, biasing along the Cartesian projection 
$(Q\sin\theta\cos\phi,\, Q\sin\theta\sin\phi)$ might seem to work well initially, but only up to the equatorial line,  where boats and twisted boats conformers are located. There, the bias force is perpendicular to the puckering sphere surface and only promotes ring expansion/contraction, breaking the ergodic sampling. The right panel shows the histogram of $Q$ and $q_r=|(q_x,q_y)|$ sampled during a metadynamics run that uses  $(q_x,q_y)$ as CVs. The algorithm becomes stuck stretching the ring, as indicated by the strong correlation between $q_r$ and the ring deformation $Q$, and is unable to leave the northern (chair-like) hemisphere. Using the Cartesian projections as CVs, the free energy estimates of accessible conformers are heavily biased \cite{sega2009calculation}. 
\end{textbox}

\subsection{Free-Energy Estimators for Metadynamics and Well-Tempered Metadynamics}

The bias potential accumulated during a MetaD or WTMetaD simulation modifies the underlying probability distribution, so that direct estimates of the free energy $F(\xi)$ from the bias $V(\xi,t)$ are inherently time-dependent. Several formulations have been proposed to obtain \emph{time-independent free-energy estimators}, which recover $F(\xi)$ or the distribution of \emph{other} observables from a trajectory sampled under the effect of a time-dependent bias \cite{metaD_reweight,wt_metaD_2,Marinova2019,giberti2019iterative,Ono2020MetaD}.

\subsubsection{Tiwary–Parrinello Time-Independent Estimator}
\input{sections/tiwary_reweighting}

\subsubsection{Bonomi–Barducci–Parrinello Reweighting.}
Bonomi et al. introduce a simple, general reweighting scheme for WTMetaD that recovers unbiased Boltzmann statistics of any observable—starting from the same key identity defining $P(\mathbf r,t)$ (Eq. \ref{eq:metad_sampled_prob}), the definition of $c(t)$ (Eq. \ref{eq:coft}) and the long-time limit of the bias constructed with WTmetaD: $V(\xi,t\to\infty)=-\Delta T/(\Delta T+T)\,F(\xi)$, Bonomi et al. develop a reweighting approach that circumvents the need of computing $c(t)$. By differentiating Eq. \ref{eq:metad_sampled_prob} for small time intervals $\Delta t$,  an evolution equation that eliminates $c(t)$ is derived: 
\begin{equation}
p(\mathbf r,t+\Delta t)=e^{-\beta\,[\dot V(\xi(\mathbf r),t)-\langle \dot V(\xi,t)\rangle]\Delta t}\,p(\mathbf r,t)
\label{eq:propagator}
\end{equation}
where $\dot c(t)=-\langle \dot V(\xi,t)\rangle$ and the average is over the biased distribution at time $t$. For WTMetaD with Gaussian depositions, these results lead to a practical reweighting algorithm based on three steps. (i) Accumulate a joint histogram $N_t(\xi,f)$ for a target variable $f(\mathbf r)$ between Gaussian updates; (ii) at each update, compute $\dot V$ and $\dot c$ using the current accumulated histogram, and evolve $N_t$ using Eq. \ref{eq:propagator}; (iii) reconstruct the unbiased distribution of $f(\mathbf{r})$ as
\begin{equation}
p(f)=\frac{\sum_{\xi} e^{+\beta V(\xi,t)}\,N_t(\xi,f)}{\sum_{\xi,f} e^{+\beta V(\xi,t)}\,N_t(\xi,f)}.
\end{equation}
The method is lightweight as it does not require any a posteriori calculation of the total energy, it works in post-processing or (in principle) on-the-fly, and it converges efficiently as shown in  Refs. \cite{gimondi2018building,Marinova2019}.


\paragraph*{Mean Force Integration (MFI)}
Extending Umbrella Integration (UI, see section \ref{sec:UmbrellaSampling}) to time-dependent biases, Mean Force Integration (MFI) provides a general estimator for history-dependent biasing schemes \cite{marinova2019time,bjola2024estimating}. Rather than computing the non-local, time-dependent bias average $c(t)$, MFI reconstructs the FES by integrating the \emph{mean force} in $\xi$, computed at each bias-update step:
\begin{equation}
\nabla F_t(s) = -\beta^{-1}\nabla\ln p_b^t(s) - \nabla V_t(s),
\end{equation}
where $p_b^t(s)$ is the biased probability density sampled while the bias $V_t(s)$ remains unchanged between updates. Averaging these mean-force estimates over successive updates and integrating numerically yields a \emph{time-independent FES}.  This formulation reveals that metadynamics, despite its adaptive nature, can be rigorously interpreted within the TI framework: the accumulated bias corresponds to an integrated mean force along the collective variables, and as such, it circumvents the need to formulate equilibration assumptions on the bias evolution. It can be applied to standard, well-tempered, adaptive-Gaussian, or transition-tempered MetaD variants. Importantly, MFI naturally supports \emph{ensemble aggregation}—it can merge sampling from multiple independent metadynamics runs without requiring continuous trajectories or recrossings \cite{marinova2019time,bjola2024estimating,serse2024unveiling}.

\subsubsection{Variationally Enhanced Sampling (VES)}
In contrast to the history-dependent or kernel-based approaches of metadynamics, VES \cite{ves_valsson2014} formulates the problem of finding the bias potential as a variational minimization of a functional of the bias potential $V(\xi)$.  Specifically, the stationary condition of the functional ensures that the biased ensemble reproduces a desired \emph{target} probability distribution $p^*(\xi)$.

The bias potential is defined as the function $V(\xi)$ that minimizes the Kullback–Leibler (KL) divergence between the sampled distribution $p_V(\xi)$ and the target distribution:

\begin{equation}
\Omega[V] = \frac{1}{\beta} \ln \left[\int d\xi e^{-\beta [F(\xi) + V(\xi)]} \right]
 \int d\xi p^*(\xi) V(\xi),
\end{equation}

where $F(\xi)$ is the underlying free energy.  Minimizing $\Omega[V]$ with respect to $V$ yields the optimal bias

\begin{equation}
V^*(\xi) = -F(\xi) - \frac{1}{\beta} \ln p^*(\xi) + \text{const.}
\end{equation}

In practice, the bias is expressed as a linear combination of basis functions (e.g., polynomials, splines, or neural-network features) with parameters optimized during the simulation through stochastic gradient descent.  This variational approach offers a systematic method for constructing bias potentials that provide direct control over the sampled distribution.  When $p^*(\xi)$ is chosen to be uniform, VES converges to a direct estimate of the free energy $F(\xi)$; when it is a tempered distribution, it behaves analogously to well-tempered metadynamics but with improved smoothness and convergence properties. A key strength of VES is how naturally it fuses with modern ML: starting from VES variational functional \cite{ves_valsson2014}, the bias can be parameterized by a neural network and optimized directly from simulation — an approach realized by Bonati et al. \cite{bonati2019neural} (“Deep-VES”), which treats the VES objective $\Omega[V]$ as a differentiable loss and updates network parameters using gradients estimated from the biased and target ensembles.

\subsubsection{On-the-fly Probability Enhanced Sampling (OPES)}
OPES \cite{invernizzi2020rethinking,invernizzi2020unified} extends metadynamics by adopting a direct probabilistic formulation. 
Instead of depositing Gaussians, OPES continuously estimates the marginal probability distribution of the CVs and updates a bias potential designed to transform the instantaneous distribution into a chosen \emph{target distribution} $\tilde p(\xi)$.  In practice, the target is often chosen to be uniform (for direct free-energy estimation) or follows a well-tempered form to strike a balance between exploration and stability. 
At every step, OPES computes the current biased histogram $p_V(\xi)$ and defines the new bias as
\begin{equation}
V(\xi) = -k_BT \ln\left[\frac{p_V(\xi)}{\tilde p(\xi)}\right].
\end{equation}
This ensures that, upon convergence, the simulation samples $p_V(\xi)=\tilde p(\xi)$.  The bias, therefore, evolves self-consistently to realise a desired stationary distribution rather than through incremental hill deposition.  In the \emph{OPES-Explore} variant, $\tilde p(\xi)$ is flat, providing a direct reconstruction of the FES; in \emph{OPES-Meta} the target adopts the same tempered form as WTMetaD, producing controlled exploration similar to well-tempered sampling but with faster convergence and reduced noise. Because OPES derives from an explicit reweighting equation, it inherits a clear statistical interpretation, and the accumulated bias approximates the free energy according to $F(\xi) = -(\gamma-1) V(\xi) + \text{const}$, analogous to WTMetaD but without relying on discrete Gaussian hills. The absence of kernel summations makes OPES computationally cheaper and smoother in high-dimensional CV spaces.  Moreover, its probabilistic update scheme naturally accommodates on-the-fly reweighting and can exploit adaptive kernel density estimators to achieve rapid convergence even in multi-dimensional landscapes. Finally, OPES offers a direct route to target distributions using CVs that can be learned, enabling an aggressive and efficient yet controlled exploration with a sound statistical reweighting procedure \cite{henin2022review,invernizzi2020unified}.